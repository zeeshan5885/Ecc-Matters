% $Id$
\documentclass[twocolumn,prd,nofootinbib]{revtex4}


\newcommand\ForInternalReference[1]{}
\newcommand\SkipForEarlyCirculation[1]{}
\newcommand\AddedResponse[1]{{\color{blue} {#1}}}
%\newcommand\SkipForEarlyCirculation[1]{#1}
\newcommand\SkipPP[1]{}
\usepackage{verbatim}
\usepackage{graphicx}
\usepackage{dcolumn}
\usepackage{bm}
\usepackage{color}
\usepackage{xspace}
\usepackage{url}
\usepackage{amsmath}
%\usepackage{adjustbox}
\usepackage{float}
\usepackage{multirow}
\usepackage{amssymb}
%
%
\usepackage{times}
%
%
%
\newcommand\optional[1]{}

%
\newcommand\E[1]{\left\langle #1\right\rangle}
\newcommand\qmstate[1]{\left|#1\right \rangle}
\newcommand\qmstateKet[1]{\left\langle#1\right|}
\newcommand\qmstateproduct[2]{\left\langle#1|#2\right\rangle}
\newcommand\qmoperatorelement[3]{\left\langle#1\left|#2\right|#3\right\rangle}
\newcommand\qmoperator[1]{{\bf #1}}
%
\newcommand\Y[1]{{{}_{#1}Y}}

\newcommand\lnL{ \ln {\cal L}}
\newcommand\lnLmarg{ \ln{\cal L}_{\rm marg}{}}
\newcommand\unit[1]{{\rm #1}}

\newcommand\rapidPEOrig{rapid\_pe1}
\newcommand\ILE{ILE}
\newcommand\editremark[1]{{\color{red} #1}}
%
%
%
\usepackage{color}
\definecolor{amber}{rgb}{1.0, 0.75, 0.0}
\definecolor{orange}{rgb}{1.0, 0.5, 0.0}
\definecolor{amaranth}{rgb}{0.9, 0.17, 0.31}
\def\fixme#1{\textcolor{red}{#1}}
\newcommand{\Richard}[1]{ {\color{blue}{#1}}}
\newcommand{\ros}[1]{ {\color{blue}{#1}}}
%

%

%
%
%
%
\graphicspath{{./figures/}}
\newcommand{\mc}{{\cal M}}
\newcommand{\Ms}{M_{\odot}}
\newcommand\LambdaTilde{\widetilde{\Lambda}}
\newcommand\DeltaLambdaTilde{\delta \widetilde{\Lambda}}
%
\def\ltsima{$\; \buildrel < \over \sim \;$}
\def\simlt{\lower.5ex\hbox{\ltsima}}
\def\gtsima{$\; \buildrel > \over \sim \;$}
\def\simgt{\lower.5ex\hbox{\gtsima}}

\newcommand\prx{Phys.~Rev.~X}
\def\aj{Astronomical Journal}                 %
\def\apj{Astrophysical Journal}                %
\def\apjl{Astrophysical Journal}             %
\def\pasj{PASJ}
\def\apjs{ApJS}              %
\def\mnras{MNRAS}            %
\def\prd{Phys.~Rev.~D}       %
\def\prl{Phys.~Rev.~Lett}    %
\def\cqg{Class.~Quant.~Grav.~}%
\def\araa{ARA\&A}             %
\def\nat{Nature}              %
\def\aap{A\&A}                %
\def\aapr{A\&A~Rev.~}    %
\def\pasp{PASP}    %
\def\sovast{Soviet Ast.}
%
%

\newcommand{\IMRPD}{\textsc{IMRPhenomD}\xspace}
\newcommand{\IMRPDT}{\textsc{IMRPhenomD\_NRTidal}\xspace}
\newcommand{\IMRP}{\textsc{IMRPhenomPv2}\xspace}
\newcommand{\SEOBP}{\textsc{SEOBNRv3}\xspace}
\newcommand{\SEOBA}{\textsc{SEOBNRv4}\xspace}
\newcommand{\SEOBAROM}{\textsc{SEOBNRv4\_ROM}\xspace}
\newcommand{\NRSur}{NRSur7dq2\xspace}
\newcommand{\TEOB}{SEOBNRv4T\xspace}
\newcommand{\Resum}{TEOBResumS\xspace}
\newcommand\RIFT{RIFT}
\newcommand{\Taylor}{TaylorF2\xspace}
\newcommand\PaperDetection{\underline{LVC-detect}\cite{DiscoveryPaper}}
\newcommand\PaperPE{\underline{LVC-PE}\cite{PEPaper}}
\newcommand\PaperTestGR{\underline{LVC-TestGR}\cite{TestingGRPaper}}
\newcommand\PaperPENRMethods{\underline{PE+NR-Methods}\cite{gwastro-PENR-Methods-Lange}}
\newcommand\PaperAstro{\underline{LVC-Astro}\cite{AstroPaper}}
\newcommand\PaperBurst{\underline{LVC-Burst}\cite{BurstPaper}}
\newcommand\PaperRates{\underline{LVC-Rates}\cite{RatesPaper}}
\newcommand\PaperStochastic{\underline{LVC-Stochastic}}
\newcommand\PaperSEOBNRvthree{\underline{LVC-SEOBNRv3}\cite{SEOBv3Paper}}

\def\RIT{Center for Computational Relativity and Gravitation, Rochester Institute of Technology, Rochester, New York 14623, USA}

\begin{document}
\renewcommand{\arraystretch}{1.5}
\title{Population Inference of Non-Spinning Eccentric Binary Black Holes}
%\author{M. Zeeshan}
\affiliation{\RIT}
%\author{R. O'Shaughnessy}

\affiliation{\RIT}
\begin{abstract}
The astrophysical mergers produce gravitational waves (GWs), which carry insightful information of their sources such as Binary Black holes (BBHs) mergers. The LIGO-VIRGO-KAGRA (LVK) detects GWs signals, and we use those measurements to infer BBHs population properties such as mass, spin, and eccentricity distribution in the universe. Although mass and spin distribution gives information of the source formation, evolution, and environment but eccentricity is the unique and measurable signature of strong events. Hence, the current parameter estimation (PE) methods of GWs signals do not include the eccentricity due to the lack of waveform with eccentricity. Therefore, for our analysis, we generated four synthetic populations of non-spinning, non-precessing, lower mass $(10 M_\odot-50 M_\odot)$, eccentric binary black holes (EBBHs) using a modified power law model, each consisting of $100$ events with different eccentricity distribution $\sigma_\epsilon = (0.05,0.1,0.15,0.2)$. Furthermore, to compare the synthetic EBBHs population with the non-eccentric BBHs (NEBBHS), we estimate how our sources would be characterized by parameter inferences which omitted the effects of eccentricity. We apply the Markov Chain Monte Carlo (MCMC) method to constrain the model parameters: event rate $\mathcal{R}$, $\alpha$, $m_{min}$, $m_{max}$, and $\sigma_\epsilon$. We are also able to recover the injected eccentricity for each population \textbf{(need to add more specfic information after runs)}. All the tools we used in this analysis are publicly available in a GitHub repository, and one can use them to infer the synthetic population of EBBHs.
\end{abstract}
\maketitle

\section{Introduction}
Black holes are intriguing and compact objects in the universe \cite{Frolic_BH_Book_2011}. They get exciting and enlightening when they coalesce and emit gravitational waves (GWs) \cite{Frolic_BH_Book_2011, Indrajit_GW_intro_1999}. These waves carry insightful information about the BBH system, such as mass, spin, period, eccentricity, distance, location, formation, and evolution. Primarily, gravitational waves may be the only way to detect those chaotic mergers in space because there was no physical evidence before LIGO's \cite{LIGO_2015} first detection of a BBH merger named GW150914. Afterward, LIGO-VIRGO-KAGRA (LVK) \cite{LIGO_2015, VIRGO_2012, Virgo-2015,kagra-2013} are detecting gravitational waves from various compact binaries such as binary black holes (BBHs), binary neutron stars (BNS), and binary made of a neutron star and black hole. The researchers use those GWs signal to infer the intrinsic and extrinsic properties of associated binary such as mass, spin, and eccentricity.  

Mass and spin are widely used to understand a binary's evolution and formation channel, ignoring the significant parameter eccentricity. Because the existence of eccentricity makes things difficult to comprehend and challenges the current understanding of the formation, environment, and evolution of such compact objects, various scenarios may develop eccentricity in BBHs orbits, such as stellar scattering, dynamic interactions in dense environments, or interaction of the third object with a binary. Therefore, a more insightful method to understand different populations and determine how a binary may have formed is to constrain the eccentricity along with mass and spin \cite{Rod-2018, zevin-samsing-2019, samsing-2018, Rodriguez-2018, Antonini-2014}.  

Low energy and slow-forming binaries usually start with lesser eccentricities, and eventually, gravitational wave radiation circularizes their orbits \cite{Peters-1964} before entering the LVK frequency band. However, binaries formed in more violent, energetic environments, such as globular clusters, can have large eccentricities. These binaries retain measurable eccentricities after entering the LVK sensitivity band, potentially making orbital eccentricity an important parameter in determining formation channels \cite{Rod-2018,zevin-samsing-2019,samsing-2018, Rodriguez-2018,Antonini-2014}. We may also have stellar mass higher eccentric mergers at the lower frequency searches \cite{sesana-2016,chen-2017}, which can be measurable by detectors like Laser Interferometer Space Antenna (LISA) \cite{LISA-2017}. These observations would allow for long-term tracking of BBHs orbital properties, which can be used to infer the formation mechanism \cite{Breivik-2016}.

Although eccentricity is a signature that can be measured with GWs and is unique to extreme events, all the confirmed detection till now shows a nearly circular orbit in the LVK frequency band. The first potential reason for getting circular orbits is that eccentric effects are more evident in low-mass events, and current searches are more efficient for higher-mass mergers. Secondly, it may result from selection biases in the waveforms because LVK detectors only use circular waveforms for parameter estimation (PE), which better present binaries evolved in an isolated environment. Studies show that we must consider the multiple formation channels to understand the population better. Because a single channel does not contribute more than $70\%$ of the observation sample of BBHs \cite{zevin-2021}. We also have growing shreds of evidence that suggests BBHs like GW190521 may exhibit unambiguous eccentricity or faint signatures \cite{Gamba_2022_GW190521_dynamical,yumeng-2023,Isobel-2022}, which leads towards the dynamical formation of the source. So, it is critical for astrophysical implications to assess eccentricity distribution.


The sensitivity of LVK detectors is increasing with time, leading to more detection in each observing run. We will have hundreds of events in future runs, and with the upcoming O4 run, we expect detection every other day \cite{detection_rate_2016,detection_rate_2015}. 
Interestingly, the new detections lead us to the reasonable disagreement on the masses, spin, event rate, and formation scenarios \cite{LSC-BBH-2016, LSC-GW150914-2016}, which pushes researchers to develop new models \cite{Mandel-2016,marchant-2016}, and also provide more information about the observable GWs \cite{Barausse-2018, Abbot-2016-schotastic,dvorkin-2016}. Additionally, researchers are actively producing the eccentric waveform models analytically \cite{Huerta-2014} and numerically \cite{gold-2016,hinder-2010, Healy-2022, Alessandro-2022, Campanelli-2009}. Those waveforms in the bank will give us real PE with eccentricity shortly. Therefore, identifying and understanding the populations of EBBHs with a growing number of detection will give us more insightful information about the formation and evolution of massive stars from birth to death over cosmic time.
 
Recent works investigate how measurements of BBHs mass and spin distributions can elucidate the population properties of BBHs \cite{Dan_2019, Michael-2015, Mandel_2017_Errors, samsing-hamers-2019,  belczynski-2016, Colm-2017, Akinobu-2017, Michael-zevin-2017, farr-2017-nature, Richard-2017-natal-kicks, Dan-Richard-2018,Abbot-2019-pop}. Researchers also study the population of BBHs with eccentricity \cite{lower-2018-ecc-pop,Fang-2019}. However, most research is inclined towards the high mass range or real events \cite{wu-2020}. Therefore, this paper focused on the non-spinning, non-precessing, lower mass $(10 M_\odot - 50 M_\odot)$, and lesser eccentric $(0-0.2)$ merger. In addition, we will use mass ratio $q=m_1/m_2$ with condition $m_1>m_2$ and total mass $M=m_1+m_2=100M_\odot$.
We present the correlation of event rate, mass, and eccentricity distribution because studies show that binary parameters and event rate are strongly coupled. We also infer how well we can recover the mass, eccentricity, and event rate using the $100$ eccentric events. To extend our analysis, we compared constrained parameters using the EBBBhs and NEBBHs, which show a considerable difference.



We organized this paper as follows. In Sec. \ref{sec:methods}, we described the Bayesian statistical methods used to make the population inference. Briefly, we also described the volume-time estimation to accommodate the LVK sensitivity. Using a truncated normal distribution, we modified the previously constructed\cite{fishbach-2017,2018talbot_bbh_model} power law model for eccentricity. We used this model to generate a synthetic population and then made the inference using it. The Sec. \ref{sec:syn_pop} describes how we have created a synthetic population using the power law model and then added an error in each event to make the population closer to real events detectable by LVK. Secondly, we explained the scaling to remove the eccentricity from the synthetic events to compare the different constraints using EBBHs and NEBBHs populations. Sec. \ref{sec:pop_inference} discusses our results and explains how well we have constrained the parameters. Also, their accuracy and recovery comparison among EBBHs and NEBBHs. Finally, we summarize our findings in Sec. \ref{sec:conclude}.



%if any component of the binary has mass less than $30 M_\odot$ then only $10\%$ of them are able to maintain eccentricity near the last stable orbit.  \cite{mass_ecc_limit_2018}.





\section{Methods}
\label{sec:methods}

The coalescing BBH can be completely described by three intrinsic and seven extrinsic parameters. The intrinsic parameters, such as the mass of the binary component ($m_i$), spin $( \chi_i)$, and eccentricity $\epsilon$  are subject to the orbital evolution of the binary. The extrinsic parameters are orientation (orbital phase, polarization, and inclination), sky location (right ascension and declination), luminosity distance, and coalescence time depend on the observer.



\subsection{Hierarchical Bayesian Modeling (HBM)}


We use HBM to constrain the population models with gravitational wave signals. In HBM, we have the N number of discrete detections. Those detections provide merger data denoted as $d_1,d_2,d_3,...,d_N$ where each $d_i$ shows a BBH merger. Each $d_i$ has mass, spin, and eccentricity properties. These properties, often called parameters, are denoted by $\lambda_1,\lambda_2,\lambda_3,...,\lambda_i$. Each parameter has its uncertainty, and we express it by the probability of the data given the parameter value. We also refer to it as the likelihood function $\mathcal{L}(\lambda)=p(d|\lambda)$ of one event. To calculate likelihood, given value of $\lambda$ is obtained from a mathematical model commonly known as a waveform . Once we have a likelihood function, you may use a uniform prior or any informative prior to find a posterior probability using the Bayes theorem as given in Eq. \ref{eq:Bayes_ind}

\begin{equation}
\label{eq:Bayes_ind}    
p(\lambda|d) \propto p(d|\lambda) p(\lambda).
\end{equation}

This posterior probability will constrain the properties of each binary, such as mass, spin, and eccentricity. We may infer those parameters using rapid parameter inference on gravitational wave sources via iterative fitting (RIFT): an open source code for parameter estimation (PE) of the binary sources \cite{rift_2018}.



%\begin{itemize}
%    \item We have N discrete detection.
%    \item each detection has data $d_1,d_2,d_3,...,d_N$
%    \item each data point $d_N$ has properties denoted by $\lambda_1,\lambda_2,...,\lambda_i$
%    \item  $ \lambda_i$ are parameters such as mass, spin, eccentricity, and location.
%    \item Each parameter has its uncertainty.
%    \item This uncertainty is described by the probability of the data given the parameter value. We call it the likelihood function of one single event. $\mathcal{L}(\lambda) = p(d|\lambda)$
%    \item the waveform gives the parameter value.
%    \item waveform is a mathematical function.
%    \item Once you have a likelihood function, you can use prior( usually it's uniform: which gives equal probability to each event) to find the posterior. $p(\lambda|d) \propto p(d|\lambda) p(\lambda)$  
    
%\end{itemize}


\subsection{Population Inference}

Now having the PE of individual sources, we will follow the Bayesian framework for population inference. The likelihood of a population parameter $\Lambda$, also considered as uncertainty in $\Lambda$ is equivalent to the probability of the individual sources given the population parameter $\Lambda$ is written as follows

\begin{equation}
\label{eq:likelihood_pop}    
\mathcal{L}(\Lambda)\equiv p(d_1,d_2,d_3,...,d_N|\Lambda).
\end{equation}

 

We will use likelihood provided in Eq. \ref{eq:likelihood_pop} in the Bayes theorem defined in Eq. \ref{eq:Bayes} to find posterior probability.

\begin{equation}
\label{eq:Bayes}    
p(\Lambda|d_1,d_2,...,d_N)= \frac{p(\Lambda)p(d_1,d_2,...,d_N|\Lambda)}{p(d_1,d_2,...,d_N)},
\end{equation}
%
where $p(\Lambda|d_1,d_2,d_3,...,d_N)$ is posterior, $p(\Lambda)$ is prior, $p(d_1,d_2,d_3,...,d_N)$ is normalization constant or also known as evidence.

To conduct our analysis for mass and eccentricity distribution, we will use the inhomogeneous Poisson process scaled by rate $\mathcal{R} = \frac{dN}{dtdV_c}$ and parameterize by $\Lambda$ to find the likelihood  $\mathcal{L}(\mathcal{R},\Lambda)\equiv p(D|\mathcal{R},\Lambda)$ of an astrophysical population given the merger rate and parameter $\Lambda$. 

\begin{equation}
\label{eq: likelihood}
\mathcal{L}(\mathcal{R},\Lambda) \propto e^{-\mu(\mathcal{R},\Lambda)}\prod_{n=1}^N\int d\lambda \ell_n(\lambda) \mathcal{R} p(\lambda|\Lambda),
\end{equation}
%
where $\mu(\mathcal{R},\Lambda)$ is the expected number of detection under the given population parametrization $\Lambda$ with the overall rate $\mathcal{R}$. $\ell_n(\lambda)=p(d_n|\lambda)$ is the likelihood of the data $d_n$ given binary parameter.
Finally, we will get our posterior as $p(\mathcal{R},\Lambda | D)\propto p(\mathcal{R},\Lambda)  \mathcal{L}(\mathcal{R},\Lambda)$ by choosing a prior $p(\mathcal{R},\Lambda)$.

These calculations are analytically intractable and must be performed numerically. Specifically, we will use Goodman and Weare's affine invariant Markov chain Monte Carlo (MCMC) \cite{mcmc_paper} to find the posterior distribution of population parameters. This method draws samples from the targeted distribution for $\Lambda$, in our case, it's the power law model given in Eq. \ref{eq:plawg}, then compares it with the given data (collection of individual events) and stores the best-fit sample. We may iterate this as we need and store multiple sample values untill they converge. 
The specific implementation we use is a Python package called EMCEE \cite{emcee_paper}.


\subsection{Volume Time (VT) Estimation}

To make our study realistic, we include the sensitivity of the LVK instruments. This sensitivity is defined by time-volume to which a census of gravitational wave events is sensitive: inferring the product $VT$.  In this expression, $V$ is the characteristic volume with units $Gpc^{3}$, which refers to the possible detection region in the sky for the LVK \cite{Volume_1993}, and $T$ is the time duration of making observations at this sensitivity.  In practice, $VT$ reflects a suitable time-averaged or cumulative sensitivity, as the true network and sensitivity varies over time.
Existing LVK instruments' sensitivity depends primarily on the mass and to a lesser extent on binary spin and (if present) modest eccentricity.  Since we neglect spin in this work, we assume the network will have the same  VT versus mass as was previously estimated  \cite{Dan_2019} for non-spinning, non-eccentric, and non-precessing binaries. Hence, we briefly explain the calculations, see \cite{Dan_2019} for details.

The Eq. \ref{eq:volume} calculates the orientation averaged sensitive volume \cite{Abbott_2016,richard2010volume}

\begin{equation}
\label{eq:volume}
V(\lambda) = \int P((<D(z))/D_n(\lambda))\frac{dV_c}{dz}\frac{dz}{1+z}
\end{equation}    
where $D(z)$ is the luminosity distance for redshift $z$, $V_c$ is the comoving volume. Finally, to compute the average number of detection, we will use Eq. \ref{eq:mu} and keep in mind that the merger will be accepted only if the signal-to-noise ratio exceeds 8 \cite{SNR_2010}.

\begin{equation}
\label{eq:mu}
  \mu(\mathcal{R},\Lambda) = \int(VT)\lambda \mathcal{R}p(\lambda|\Lambda)d\lambda ,
\end{equation}
where $p(\lambda|\Lambda)$ is the probability density function for a random binary in the universe to have intrinsic parameter $\lambda$. Keep in mind that $\lambda$ is equal to all intrinsic and extrinsic parameters.


\subsection{Power law Model}
There are various weak and pure phenomenological population models proposed in previous studies \cite{2016PRXAbbot_BBH_model,2017FishBach_BBH_model,2018talbot_bbh_model}. However, our analysis used the pure truncated power law defined in \cite{2016PRXAbbot_BBH_model,2017FishBach_BBH_model} and Gaussian eccentricity. This model computes the intrinsic probability of $m_1$, $m_2$, and $\epsilon$.  
We also assume that non-zero probability density only exists for $m_{min}\leq m_2 \leq m_1 \leq m_{max}$ and for total mass $M_{max}=m_1+m_2 = 200 M_\odot$. The condition of the total mass is only because of the limitations of the detectors towards the higher masses \cite{2016PRXAbbot_BBH_model}. The generalized form of the truncated power-law model with parameters $\Lambda \equiv  (\alpha, \mathcal{R}, k_m, m_{min}, m_{max}, \sigma_\epsilon, M_{max})$ and random variable $m_1$, $m_2$, and $\epsilon$ has the functional form in Eq. \ref{eq:plawg} within provided mass limit.


\begin{align}
\label{eq:plawg}
p(m_1,m_2,\epsilon) = &C(\alpha,k_m,m_{min},m_{max},M_{max},\epsilon)  
\nonumber \\ & \sqrt{\frac{2}{\pi}} \frac{(m_2/m_1)^{k_m} m_1^{-\alpha} e^{-(\epsilon/\sqrt{2}\sigma_\epsilon)^2}}{(m_1-m_{min})\sigma_\epsilon},
\end{align}
%
where $\alpha$ is the power law index, $\mathcal{R}$ is the merger rate, $m_{min}, m_{max}$ are the minimum and maximum masses of the binary components in the population, and $\sigma_e$ is the orbital eccentricity distribution. The Eq. \ref{eq:plawg} represent a truncated power law for primary mass $m_1$ with index $-\alpha$ and conditional power law distribution $p(m_2|m_1)$ for secondary mass $m_2$ using simple power law, and Gaussian distribution for orbital eccentricity $e$. 
For our analysis, we defined a constant of integration equal to $\int_V dm_1 dm_2 d\epsilon p(m_1,m_2,\epsilon) = 1$.  Our detectors are sensitive to high-mass BBHs, particularly $M_{max}\geq 200 M_\odot$. Therefore, we will use $k_m=0$ throughout the studies. As a result, we have our reduced form of the truncated power law in Eq. \ref{eq:plaw}

\begin{align}
\label{eq:plaw}
p(m_1,m_2,\epsilon) = \sqrt{\frac{2}{\pi}} \frac{ m_1^{-\alpha}  e^{-(\epsilon/\sqrt{2}\sigma_\epsilon)^2}}{(m_1-m_{min})\sigma_\epsilon}
\end{align}

\section{Synthetic Population}
\label{sec:syn_pop}
We have created four synthetic populations by choosing the power law parameters $\alpha = -1$, $m_{min} = 10$, $m_{max}=50$. Each population has the same values for $\alpha$, $m_{min}$, and $m_{max}$ except $\sigma_\epsilon$. We selected four values for $\sigma_\epsilon = 0.05, 0.1, 0.15, 0.2$, and  generated synthetic population containing $10000$ sources.

\subsection{Synthetic population with eccentricity}

After generating four populations, we find the probability for each event to be detected by computing the VT of each source. Finally, we did the weighting based on these VTs and randomly picked $N=100$ sources from each population to perform our analysis. Our weighted populations with $\sigma_\epsilon=0.05$ and $\sigma_\epsilon=0.2$ are shown in Fig. \ref{fig:pop3d0.05_0.2}.  The complete description of each EBBHs population is provided in Table \ref{tab:pop_prop}.

\begin{figure*}[]
\includegraphics[width=0.45\textwidth]{paper/figures/pop3d_0.05.png}
\includegraphics[width=0.45\textwidth]{paper/figures/pop3d_0.2.png}
\caption{\label{fig:pop3d0.05_0.2} Synthetic Population of EBBHs. The left figure represents the population for $\sigma_\epsilon=0.05$ and right figure represents the population for $\sigma_\epsilon=0.2$}
\end{figure*}


\begin{table*}[]
    \centering
    \begin{tabular}{c|cccccc}
        \hline \hline
        Populations & $m_{1_{min}} [M_\odot] $ & $m_{1_{max}} [M_\odot]$ & $m_{2_{min}} [M_\odot]$ & $m_{2_{max}} [M_\odot]$ & $\epsilon_{min}$ & $\epsilon_{max}$\\ \hline
        First  & 12.19 & 49.89 & 10.17 & 46.67 & 0.0004 & 0.0494\\ \hline
        Second  & 15.42 & 49.79 & 10.46 & 47.59 & 0.0003 & 0.0999\\ \hline
        Third  & 16.17 & 49.83 & 10.69 & 47.52 & 0.0002 & 0.148\\ \hline
        Forth  & 12.52 & 49.89 & 10.08 & 49.51 & 0.0012 & 0.1984\\ \hline
    \end{tabular}
    \caption{Population Properties of EBBHs}
    \label{tab:pop_prop}
\end{table*}

To make our study more realistic, we must add the measurement error in each source.  Rather than generate synthetic gravitational wave sources and perform full Bayesian inference, following previous work \cite{Mandel_2017_Errors} we generate mock measurement errors motivated by real parameter inference investigations. The chirp mass and symmetric mass ratio are well-constrained compared to the primary and secondary masses of BBHs. Therefore, we compute them for each event in a population by using the following relations.

\begin{equation}
    M_c^T = \frac{(m_1 m_2)^{3/5}}{(m_1+m_2)^{1/5}},
\end{equation}

\begin{equation}
    \eta^T = \frac{(m_1 m_2)}{(m_1+m_2)^2},
\end{equation}
%
where $M_c^T$ and $\eta^T$ are found using the primary and secondary masses of each event generated by the power law model.
Furthermore, using the following relations, We add the measurement errors in the $M_c^T$ and $\eta^T$.

\begin{equation}
    M_c = M_c^T\left( 1+\alpha (r_0+r ) \frac{12}{\rho}\right),
\end{equation}

\begin{equation}
\eta = \eta^T\left( 1+0.03 (r_0'+r') \frac{12}{\rho}\right),   
\end{equation}
where $r_0$ and $r_0'$ are the random numbers drawn from the standard normal distribution, which will shift the mean of the $M_c$ and $\eta$ distribution with respect to $M_c^T$ and $\eta^T$. The $r$ and $r'$ are the independent and identically distributed arrays of those randomly generated numbers to spread the distribution. The measurement uncertainty is inversely proportional to signal-to-noise ratio $\rho$, drawn from the distribution $p(\rho) \propto \rho^{-4}$, which holds for isotropically distributed sources in a static universe, subject to the threshold $\rho\geq 8$ for detection. The $\alpha =0.1$ is used for the scaling inspired by analyses of mock data with the LALINFERENCE pipeline \cite{alpha_error_2015} and includes the impact of correlation with parameters describing arbitrary remnant spins.

Finally, after adding the measurement errors in the $M_c$ and $\eta$, we will convert them back to $m_1$ and $m_2$ to perform our analysis. We used the following relation for conversion, and it will provide the masses based on the condition $m_1\geq m_2$.

\begin{equation}
    m_1 = \frac{1}{2} M_c \eta^{-3/5} (1+\sqrt{\eta_v}),
\end{equation}

\begin{equation}
    m_2 = \frac{1}{2} M_c \eta^{-3/5} (1-\sqrt{\eta_v}), 
\end{equation}
where $\eta_v = 1-4\eta$, we kept the samples with non-negative values and ignored the negative samples to avoid the square root issues. 

We also added the absolute error in the eccentricity by using the truncated normal distribution scaling at $0.05$ to keep the eccentricity positive.
\textbf{this is important - refer to other papers.  THIS IS REASONABLE FOR LOW MASS BUT VERY OPTIMISTIC FOR HIGH MASS}
STRETCH GOAL: what if you ran again, but with large error?








\subsection{Synthetic population without eccentricity}

To compare the synthetic EBBHs population with the non-eccentric BBHs (NEBBHS), we estimate how our sources would be characterized by parameter inferences which omitted the effects of eccentricity.  Following \cite{2021_scaling_paper}, the following effective chirp mass is well-constrained by observations dominated by the inspiral of a slightly eccentric binary:
\begin{align}
\label{eq:scaling}
M^{ecc} = \frac{M}{(1-\frac{157}{24}\epsilon^2)^{3/5}}
\end{align}
%
Our ansatz for source identification and characterization is that the best-fitting parameters and posteriors are directly related to the true posteriors, except that the recovered chirp mass is given by  Eq. \ref{eq:scaling}.  This process removes the eccentric component from the population only by scaling the masses and omitting the eccentricity. 





%\begin{table*}[]
%    \centering
%    \begin{tabular}{c|cccc}
%        \hline \hline
%        Populations & $m_{1_{min}} [M_\odot] $ & $m_{1_{max}} [M_\odot]$ & $m_{2_{min}} [M_\odot]$ & $m_{_2{max}} [M_\odot]$ \\ \hline
%        First & 12.09 & 49.6 & 10.17 & 46.27 \\ \hline
%        Second & 15.26 & 49.77 & 10.3 & 47.56 \\ \hline
%        Third & 15.96 & 49.5 & 10.62 & 46.60  \\ \hline
%        Forth & 11.86 & 49.76 & 9.67 & 46.99  \\ \hline
%    \end{tabular}
%    \caption{Population Properties of NEBBHs}
    \label{tab:popscl_prop}
%\end{table*}
 

\begin{figure*}

\includegraphics[width=0.45\textwidth]{paper/figures/pop2d_0.05.png}
\includegraphics[width=0.45\textwidth]{paper/figures/pop2d_0.2.png}
\caption{\label{fig:pop2d_0.05_0.2} The left figure shows the primary mass vs secondary mass of the EBBHs and NEBBHs with $\sigma_\epsilon =0.05$. The right figure shows the same plot but for $\sigma_\epsilon=0.2$} 

\end{figure*}




%To get the non-eccentric population, we apply the Eq. \ref{eq:scaling} on the population given in the table. \ref{tab:pop_prop}. After applying the scaling equation, we get the new populations with scaled masses, giving us the new minimum and maximum values for the primary and secondary masses given in table \ref{tab:popscl_prop}.
   


The significant effect after scaling is the mass shift, which can be observed in Fig. \ref{fig:pop2d_0.05_0.2}. In this figure, the left-hand side shows the mass shift of the first population generated with smaller $\sigma_\epsilon =0.05$, which leads to a lesser mass shift. However, on the right hand of Fig. \ref{fig:pop2d_0.05_0.2}, one can see the significant mass shift after removing the eccentric component.  This also reflects that we may miss the various sources by ignoring the effect of eccentricity and we notice that sources start missing, which has eccentricity $\epsilon>0.38$.


                            

\section{Population Inference}
\label{sec:pop_inference}
%\renewcommand{\arraystretch}{2}
\begin{table*}[]
    \centering
    \begin{tabular}{c|ccccc}
        \hline \hline
       Inference & $log_{10}(\frac{\mathcal{R}}{Gpc^{-3}yr^-1})$ & $\alpha$ & $m_{min} [M_\odot] $ & $m_{max} [M_\odot]$ & $\sigma_\epsilon$ \\ \hline
      First& $1.42^{+0.58}_{-2.64}$ & $-0.88^{+3.17}_{-1.89}$ & $8.36^{+7.49}_{-3.51}$ & $58.44^{+10.21}_{-24.31}$ & $0.30^{+0.54}_{-0.25}$ \\ \hline
      Second & $1.36^{+0.63}_{-2.65}$ & $-0.93^{+3.29}_{-1.92}$ & $8.71^{+7.36}_{-3.81}$ & $57.84^{+10.81}_{-25.53}$ & $0.33^{+0.51}_{-0.25}$ \\ \hline
      Third & $1.39^{+0.58}_{-2.69}$ & $-0.69^{+2.98}_{-1.95}$ & $9.21^{+6.31}_{-3.99}$ & $57.85^{+10.73}_{-27.04}$ & $0.33^{+0.51}_{-0.23}$  \\ \hline
      Forth & $1.27^{+0.72}_{-2.68}$ & $-0.90^{+3.30}_{-1.98}$ & $8.63^{+7.85}_{-3.72}$ & $57.76^{+11.03}_{-25.82}$ & $0.37^{+0.47}_{-0.23}$  \\ \hline
    \end{tabular}
    \caption{Population Inference for EBBHs}
    \label{tab:inference_EBBHS}
\end{table*}


\begin{table*}[]
    \centering
    \begin{tabular}{c|cccc}
        \hline \hline
        Inference & $log_{10}(\frac{\mathcal{R}}{Gpc^{-3}yr^-1})$ & $\alpha$ & $m_{min} [M_\odot] $ & $m_{max} [M_\odot]$ \\ \hline
      First & $1.85^{+0.14}_{-1.23}$ & $-1.37^{+1.43}_{-0.94}$ & $7.54^{+3.68}_{-3.04}$ & $63.29^{+8.00}_{-13.08}$  \\ \hline
      Second & $1.83^{+0.14}_{-1.38}$ & $-1.12^{+1.47}_{-0.91}$ & $7.69^{+3.88}_{-3.15}$ & $61.91^{+8.59}_{-13.87}$  \\ \hline
      Third & $1.84^{+0.15}_{-1.55}$ & $-0.38^{+1.03}_{-0.96}$ & $8.07^{+3.44}_{-3.27}$ & $61.70^{+8.49}_{-24.26}$   \\ \hline
      Forth & $1.86^{+0.16}_{-1.85}$ & $-0.31^{+1.17}_{-1.14}$ & $6.99^{+8.77}_{-2.58}$ & $61.15^{+8.72}_{-16.84}$  \\ \hline
    \end{tabular}
    \caption{Population Inference for NEBBHs}
    \label{tab:inference_NEBBHS}
\end{table*}


Figure A shows the results of population inference under the most conservative case with $\sigma_\epsilon=0.05$.  In this figure, the ZZ color shows the results of QQ.
%
Most notably, this figure shows that even in the limit of extremely small ecctricity, the impact of eccentricity can be measured from a large population of observations, comparable to the current size.  This result depends of course on our assumption that all binaries are equally likely to have eccentricity, and that our measurement uncertainty in eccentricity is both indpeendent of mass and fairly small, both optimisitic assumptions.  In fact, the overall uncertainty in $\sigma_\epsilon$ is expected: given 100 events, each with a measurement error of $0.05$, we expect $\sigma_\epsilon$ to be measured to of order $0.05/\sqrt{N}$ (handwavy -- really a chisquared distribution,etc).
SUSPECT THAT IF YOU MAKE THIS JUST 3X bigger, STILL GET RESULT -- you will almost always get this result

Also noteworthy is the impact of neglecting eccentricity.  If eccentricity is neglected, then the population inference is biased away from true values: the black contours in the first column do not include the true values for $m_{max}, m_{min}$ example.



Figure B shows the results of population inference under the most optimistic case with $\sigma_\epsilon=0.2$.  In this figure, the ZZ color shows the results of QQ.


we can discuss the corner plots here.

TAKEAWAYS

* Eccentricity can be measured

* If not measured, biases




\begin{figure*}

\includegraphics[width=0.95\textwidth]{paper/figures/cor_0.05.png}
\caption{\label{fig:pop3d05}\textbf{Corner Plots for EBBHs and BBHS for $\sigma_\epsilon=0.05$}}

\end{figure*}
Same population.
\begin{figure*}

\includegraphics[width=0.95\textwidth]{paper/figures/cor_0.2.png}
\caption{\label{fig:pop3d05}\textbf{Corner Plots for EBBHs and BBHs for $\sigma_\epsilon=0.2$}}

\end{figure*}

Need to write it later. 







% \begin{align} \label{eq:strain_mode}
% h(t,\vartheta,\phi;\bm{\lambda}) = 
% \sum_{\ell=2}^{\infty} \sum_{m=-\ell}^{\ell} \frac{D_{\rm ref}}{D} h^{\ell m}(t;\bm{\lambda}) \Y{-2}_{\ell m} \left(\vartheta, \phi \right) \, ,
% \end{align}


% \begin{widetext}
% \begin{align}
% \ln {\cal L}(\bm{\lambda}, \theta) 
% &= (D_{\rm ref}/D) \text{Re} \sum_k \sum_{\ell m}(F_k \Y{-2}_{\ell m})^* Q_{k,lm}(\bm{\lambda},t_k)\nonumber \\
% &   -\frac{(D_{\rm ref}/D)^2}{4}\sum_k \sum_{\ell m \ell' m'}
% \left[
% {
% |F_k|^2 [\Y{-2}_{\ell m}]^*\Y{-2}_{\ell'm'} U_{k,\ell m,\ell' m'}(\bm{\lambda})
% }
% % \right. \nonumber \\ & \left.
%  {
% +  \text{Re} \left( F_k^2 \Y{-2}_{\ell m} \Y{-2}_{\ell'm'} V_{k,\ell m,\ell'm'} \right)
% }
% \right]
% \label{eq:def:lnL:Decomposed}
% \end{align}
% \end{widetext}

% \begin{eqnarray}
% {\cal L}_{\rm margT} \equiv  \int {\cal L} \frac{dt}{T}
% \label{eq:lnL:tmarg}
% \end{eqnarray}












% \begin{table*}
% \begin{tabular}{lrr|ccccc|rr}
% Version & srate & modes & $\tau_{start}$ & $\tau_{setup}$ & $\tau_{ad}$ & $\tau_{it,like}$ &$\tau_{it,rest}$ &
% $\frac{T_{ILE}}{N_{eval}}$ & GPU \\  %\hline 
%   &   Hz & m & sec & sec & & $\mu$sec & $\mu$sec  &sec  & use  \%\\ \hline 
% % ~/parse_report.sh profile_nogpu_pcdev13.log | more
% CPU & 16384 & $\pm 2 $ & 20 & 2.4 &&540 & 20 &  690  \\ 
%        & 4096 & $\pm 2 $ &   20  &&&& 20 \\ \hline
% % ./parse_report.sh 20190130-profile_nogpu_HM_pcdev13.log  | more
% % /parse_report.sh ./profile_nogpu_HM_lowres_pcdev13.log 
% %    setup time: PrecomputeLikelihoodTerms, includes waveform generation. 
% %   evaluation: FactoredLogLikelihodTimeMarginalized Divide by actual number of calls, since not a block!
% %    
% CPU & 16384 & $\pm 2,\pm 1 $ & 20 & 1.5 && 680 & 20 &  1060  \\ 
%        & 4096 & $ \pm 2, \pm 1 $ &   20 &&&& 20  \\ \hline

%GPU (a) & 16384 & $\pm 2 $  & 20  & & && & 270 \\
%            & 4096 &$\pm 2 $  &  20 &  & & & & 45 \\ \hline

%GPU (b) & 16384 & $\pm 2$ & 20  & 1.8 & 1 & 0.85& 20 &28 & 15\\
%        & 4096 & $\pm 2$  & 20 & $1.2 $ &  1  & 0.75 & 20  & 25\\ \hline

% GPU (b) & 16384 & $\pm 2, \pm 1$ & 20 & 1 && 4.2 & 20  & 38  \\
%        & 4096 & $\pm 2, \pm 1$ & 20 & 1&& 2.5  & 20 & 35 & \\ \hline

% GPU (c) & 16384 & $\pm 2 $  & 20  &6  & & 18&  58&160 &  \\
%             & 4096 &$\pm 2 $  &  20 & 3.7 & & 11  & 58  & 140 & $\simeq 50$ \\
% \end{tabular}
% Compute node at LIGO-WA
% \caption{\label{tab:CostBreakdown}\textbf{Profiling performance: Binary black holes}: Evaluation costs for the
%   marginalized likelihood on default
%   hardware, for a two-mode system $(l,m)=\pm 2$ analyzing $T=8\unit{s}$ of data with a massive binary black hole
%   $m_1=35 M_\odot,M_2=30 M_\odot$.  The last column indicates peak GPU utilization.
% }
%\end{table*}










% \subsection{Binary black hole analysis}
% \label{sec:sub:BBHFull}

% \begin{figure*}

% \includegraphics[width=0.45\textwidth]{figures/bbh_zerospin_m1_m2.png}
% \includegraphics[width=0.45\textwidth]{figures/bbh_zerospin_m1_cum.png}
% % python plot_mean_variance.py --convergence-file 20190203-bbh-zerospin-batch_gpu_lowlatency_meanVar.dat  --output bbh_zerospin_lnL_meanVar
% \includegraphics[width=0.45\textwidth]{figures/bbh_zerospin_lnL_meanVar.png}
% % python plot_convergence.py --convergence-file 20190203-bbh-zerospin-batch_gpu_lowlatency.dat
% \includegraphics[width=0.45\textwidth]{figures/bbh_zerospin_lnL_converge.png}
% \caption{\label{fig:BBH:MultiIterate}\textbf{Convergence of BBH analysis: Zero spin}: Results for marginal posterior distributions
%   of our fiducial synthetic binary black hole.  Solid contours show credible intervals; solid one-dimensional distributions
%   show marginal CDFs and PDFs for the corresponding variable; and colored points indicate the location $\bm{\lambda}$ and
%   value of the underlying marginalized likelihood evaluations.   
% %The corresponding dotted curves show an analysis using   only the $m=\pm 2$ modes \editremark{perform both} 
% \emph{Left panel } Posterior distribution
%   over  $\mc$ and
%   $\delta=(m_1-m_2)/M$.    \emph{Right panel}: Marginal 1d CDFs of $\mc$, showing convergence.
% \emph{Bottom left}: Mean and variance of  \AddedResponse{the array $\ln{\cal L}_{\rm marg}(\bm{\lambda}_j)$  for
% $j=1,2,\ldots N_{\rm eval}$ indexing all candidate sets of intrinsic parameters $\bm{\lambda}_j$ performed in that iteration},  showing that after the
% first iteration the
% candidate points are consistent with the posterior (i.e., no proposed point has very low $\ln {\cal L}_{\rm marg}$).
% \emph{Bottom right panel}: The estimated evidence $Z = \int d\bm{\lambda} {\cal L}_{\rm marg}$ versus iteration number.  As systematic fitting error dominates our
% error budget, Monte Carlo error is not shown.
% }
% \end{figure*}














\section{Conclusions}
\label{sec:conclude}
In Progress, 




* summarize your calculations

* summarize key takeaways

** we WILL find this, measuremeent error can't be too large/infinite!

* contrast /compare with other papers 

* some comments on what it all means for the big picture (dynamical evolution tests for example could be done)














\begin{acknowledgements}
We thank our anonymous referee for the helpful feedback.

\end{acknowledgements}


\appendix
In Progress,




%\bibstyle{unsrt}
\bibliography{paperexport,LIGO-publications}
\end{document}
